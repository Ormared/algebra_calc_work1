\documentclass[a4paper,11pt]{article}
% language packages
\usepackage[T1,T2A]{fontenc}
\usepackage[utf8]{inputenc}
\usepackage[english,ukrainian]{babel}

\usepackage{scratch3}
\usepackage{enumitem}
\usepackage{graphicx,tikz}
\usepackage[all]{xy}
\usepackage{amsmath,amssymb,ntheorem}
\usepackage{mathtools}
% \usepackage{linalgjh}
% \texttt{gauss.sty}
\usepackage{gauss}
\usepackage{ntheorem}
\theoremstyle{break}
\theorembodyfont{\upshape}
\newtheorem{example}{Example}

\title{Розрахункова Частина}
\author{Демідов Максим}
\date{Варіант-8}


\begin{document}
\maketitle
\section{Завдання 1. Для даного визначника:}
\begin{enumerate}[label=(\alph*)]
    \item знайти мінори та алгебричні доповнення елементів;
    \item обчислити визначник, розкладаючи його за елементами i-го рядка;
    \item обчислити визначник, розкладаючи за елементами j-го стовпця;
    \item обчислити визначник, утворивши попередньо нулі в і-му рядку.
\end{enumerate}
\begin{gmatrix}
    0 & 4 & 1 & 1 \\
   -4 & 2 & 1 & 3 \\
    0 & 1 & 2 & -2 \\
    1 & 3 & 4 & -3 
\end{gmatrix}
, $ i = 4 $, j = 3 \\
\begin{enumerate}
    \item Мінори та алгебричні доповнення елементів\\
        \begin{gmatrix}[b]
            0 & 4 & 1 & 1 \\
           -4 & 2 & 1 & 3 \\
            0 & 1 & 2 & -2 \\
            1 & 3 & 4 & -3 
            \rowops
            \swap{0}{1}
            \mult{1}{\cdot -1}
        \end{gmatrix}

        \begin{gmatrix}
            -4 & 2 & 1 & 3 \\
            0 & -4 & -1 & -1 \\
            0 & 1 & 2 & -2 \\
            1 & 3 & 4 & -3
            \rowops
            \add[\cdot (\frac{1}{4})]{0}{3}
        \end{gmatrix}       

\end{enumerate}

\end{document}
